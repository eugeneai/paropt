\documentclass[10pt]{beamer}
\usepackage{amsmath,amssymb,longtable,hhline}
\usepackage{mathrsfs}
\usepackage{xcolor}
\usepackage{listings}

\usetheme{Warsaw}
\usecolortheme{crane}
\beamertemplatenavigationsymbolsempty

\usepackage{iftex,ifxetex}
\ifPDFTeX
  \usepackage[utf8]{inputenc}
  \usepackage[T1]{fontenc}
  \usepackage[russian]{babel}
  \usepackage{lmodern}
  \usefonttheme{serif}
\else
  \ifluatex
    \usepackage{unicode-math}
    \defaultfontfeatures{Ligatures=TeX,Numbers=OldStyle}
    \setmathfont{Latin Modern Math}
    \setsansfont{Linux Biolinum O}
    \usefonttheme{professionalfonts}
    \setmathfont[
        Ligatures=TeX,
        Scale=MatchLowercase,
        math-style=upright,
        vargreek-shape=unicode
        ]{euler.otf}
  \fi
\fi

%\useoutertheme{split}
%\useinnertheme{rounded}
\setbeamertemplate{background canvas}[vertical shading][bottom=white!80!cyan!20,top=cyan!10]
%\setbeamertemplate{sidebar canvas left}[horizontal shading][left=white!40!black,right=black]

\graphicspath{{pics/}}

\begin{document}
\title[КОМПЬЮТЕРНАЯ АЛГЕБРА В ОПТИМИЗАЦИИ]{ПРИМЕНЕНИЕ КОМПЬЮТЕРНОЙ АЛГЕБРЫ\\
В РЕАЛИЗАЦИИ АЛГОРИТМОВ УЛУЧШЕНИЯ}
\author{Черкашин Е.А., Бадмацыренова С.Б.}
\institute[ИДСТУ СО РАН, ИРНИТУ, ИМЭИ ИГУ]{\normalsize  ИДСТУ СО РАН, ИРНИТУ, ИМЭИ ИГУ}
\date[2015]{{}\\[1.5cm]
<<Винеровские чтения --- 2015>>\\
16 -- 18 апреля 2015 г.,
Иркутск
}
%\date{\today}
\maketitle

% ----------------------------------------------------------------
\begin{frame}[plain]{\textbf{Вектор-функция} }

%\begin{block}{}
  \begin{columns}[t]
    \begin{column}{0.65\textwidth}
      \def\xyz{$x_1,x_2,x_3$} \def\yaw{$x_6$} \def\pitch{$x_5$}
      \def\roll{$x_4$} \def\lift{$x_9$} \def\down{$\quad\! x_7$}
      \def\thrust{$u_1$} \def\rudder{$u_5$} \def\drag{$x_8$}
      \def\flaps{$u_2$} \def\aeleron{$u_3$} \def\engine{}
      \def\elevator{$u_4$} \def\svgwidth{\columnwidth}
      \input{pics/Cessna-Plane.pdf_tex}
    \end{column}
    \begin{column}{0.35\textwidth}
      \begin{block}{Состояние, управление}%
        \begin{align*}
          &\vec{x}=\mathbf{x}=\langle x_1,x_2,\ldots,x_n\rangle,\\
          &n=9,\\
          &\vec{u}=\mathbf{u}=\langle u_1,u_2,\ldots,u_m\rangle,\\
          &m=5,\\
          &\vec{f}=\mathbf{f}.
        \end{align*}
      \end{block}
    \end{column}
  \end{columns}
%\end{block}
\begin{block}{Уравнение движения}
  \begin{align*}
&\mathbf{x}(t)=\mathbf{f}(t,\mathbf{x}(t),\mathbf{u}(t)),\quad t \in T=[t_0,t_1], \\
&\mathbf{x}(t_0)=\mathbf{x}_0,\quad \mathbf{x}(t)\in \mathbb{R}^n,\; \mathbf{u}(t) \in \mathbb{R}^m,\quad t\in T, \\
&I(\mathbf{x},\mathbf{u})=\int\limits_{t_0}^{t_1}f^0(t,\mathbf{x}(t),\mathbf{u}(t)) \to \min.
  \end{align*}
\end{block}
\end{frame}
\begin{frame}{\textbf{Постановка задачи} }
\begin{block}{Уравнение движения (дискретный вариант)}
  \begin{align}
&\mathbf{x}(t+1)=\mathbf{f}(t,\mathbf{x}(t),\mathbf{u}(t)),\quad t \in T=\{t_0,t_0+1,...,t_1\},
	\label{eq100} \\
&\mathbf{x}(t_0)=\mathbf{x}_0,\quad \mathbf{x}(t)\in \mathbb{R}^n,\; \mathbf{u}(t) \in \mathbb{R}^m,\quad t\in T,\label{eq100a} \\
&I(\mathbf{x},\mathbf{u})=F(\mathbf{x}(t_1))+ \sum_{t_0}^{t_1-1}f^0(t,\mathbf{x}(t),\mathbf{u}(t)) \to \min.
  \label{eq103}
  \end{align}
\end{block}
\begin{block}{Задача улучшения}
  Пусть заданы управление $\mathbf{u}^I(t)$ и соответствующее состояние $\mathbf{x}^I(t)$. Требуется найти $\mathbf{x}^{I\!I}(t),$ $\mathbf{u}^{I\!I}(t)$ такие, что
\[
I(\mathbf{x}^{I\!I},\mathbf{u}^{I\!I}) < I(\mathbf{x}^I,\mathbf{u}^I).
\]
 \end{block}
\end{frame}
% ---------------------------------------------------------
\def\H{\mathbf{H}}
\def\x{\mathbf{x}}
\def\u{\mathbf{u}}
\def\f{\mathbf{f}}
\def\F{\mathbf{F}}
\def\bpsi{\mathbf{\psi}}
% ----------------------------------------------------------------
\begin{frame}[shrink]{\textbf{Алгоритм первого порядка} }
%\begin{block}{}
\begin{enumerate}
 \item[1.] Задается начальное управление $\mathbf{u}^I(t)$, из уравнения \eqref{eq100} и условий \eqref{eq100a} определяется $\mathbf{x}^I(t)$. Вычисляется $I(\mathbf{x}^I,\mathbf{u}^I).$
 \item[2.] Из системы $\mathbf{\psi}(t)=\mathbf{H}_\mathbf{x},\;\bpsi(t_1)=-\F_\x,$ находим $\bpsi(t),$\\ где $\bpsi(t)$ --- $n$-вектор, $\H_\u(t,\x,\bpsi(t+1),\u)=\bpsi^{T}(t+1)\times\f(t,\x,\u)-f^{0}(t,\x,\u),$ производные функции $H$ по $\u$ ($\H_\u$) находятся в точке $\left(t,\x^{I}(t),\bpsi \left(t+1 \right),\u^{I}(t)\right)$. Задается параметр $\alpha$.
 \item[3.] Из системы $\x(t+1)=\f(t,\x(t),\u^{I\!I}(t)), \x(t_0)=\x_0,$ где $\u^{I\!I}(t)=\u^{I}(t)+\alpha \H_\u,$ вычисляется $\x^{I\!I}(t).$
 \item[4.] Новое управление и значение параметра $\alpha$ подсчитываются из решения задачи одномерной минимизации для функционала $I(\x^{I\!I},\u^{I\!I})\to \min\limits_{\alpha}.$
 \item[5.] Если $I\left(\x^{I\!I},\u^{I\!I}\right)\geq I\left(\x^{I},\u^{I}\right)$ (улучшение не произошло), то уменьшаем $\alpha$ и переходим к следующей итерации, начиная с пункта~3.
 \item[6.] Иначе, если $I\left(\x^{I\!I},\u^{I\!I}\right) - I\left(\x^{I},\u^{I}\right)>\epsilon$, то переходим к следующей итерации, начиная с пункта~2. Значение $\epsilon$ --- параметр точности.
\end{enumerate}
%\end{block}
\end{frame}
% ----------------------------------------------------------------

\begin{frame}[fragile]
\frametitle{Source code}

\begin{lstlisting}[caption=First C example]
int main()
{
    printf("Hello World!");
    return 0;
}
\end{lstlisting}

\begin{itemize}
  \item This one is always shown
  \item<1-> The first time (i.e. as soon as the slide loads)
  \item<2-> The second time
  \item<1-> Also the first time
  \only<1-1> {This one is shown at the first time, but it will hide soon (on the next event after the slide loads).}
\end{itemize}
\end{frame}



\begin{frame}{}
\begin{center}
\Huge\textbf{Спасибо за внимание!}
\end{center}

\end{frame}
% ----------------------------------------------------------------

\end{document}
%

%%% Local Variables:
%%% mode: latex
%%% TeX-master: t
%%% End:
