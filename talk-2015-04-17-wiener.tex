\documentclass[10pt]{beamer}
\usepackage{amsmath,amssymb,longtable,hhline}
\usepackage{mathrsfs}
\usepackage{xcolor}

\usetheme{Warsaw}
\usecolortheme{crane}

\usepackage{iftex,ifxetex}
\ifPDFTeX
  \usepackage[utf8]{inputenc}
  \usepackage[T1]{fontenc}
  \usepackage[russian]{babel}
  \usepackage{lmodern}
  \usefonttheme{serif}
\else
  \ifluatex
    \usepackage{unicode-math}
    \defaultfontfeatures{Ligatures=TeX,Numbers=OldStyle}
    \setmathfont{Latin Modern Math}
    \setsansfont{Linux Biolinum O}
    \usefonttheme{professionalfonts}
    \setmathfont[
        Ligatures=TeX,
        Scale=MatchLowercase,
        math-style=upright,
        vargreek-shape=unicode
        ]{euler.otf}
  \fi
\fi

\graphicspath{{pics/}}

\begin{document}
\title{ПРИМЕНЕНИЕ КОМПЬЮТЕРНОЙ АЛГЕБРЫ\\
В РЕАЛИЗАЦИИ АЛГОРИТМОВ УЛУЧШЕНИЯ}
\author{Черкашин Е.А., Бадмацыренова С.Б.}
\institute[ИДСТУ СО РАН, ИРНИТУ]{\normalsize  ИДСТУ СО РАН, ИРНИТУ}
\date[2015]{<<Винеровские чтения --- 2015>>
\\[0.3cm]
16 -- 18 апреля 2015 г.,\\
Иркутск
}
%\date{\today}
\maketitle

% ----------------------------------------------------------------
\begin{frame}{\textbf{Вектор-функция} }

%\begin{block}{}
  \begin{columns}[t]
    \begin{column}{0.75\textwidth}
      \def\xyz{$x_1,x_2,x_3$} \def\yaw{$x_6$} \def\pitch{$x_5$}
      \def\roll{$x_4$} \def\lift{$x_9$} \def\down{$\quad\! x_7$}
      \def\thrust{$u_1$} \def\rudder{$u_5$} \def\drag{$x_8$}
      \def\flaps{$u_2$} \def\aeleron{$u_3$} \def\engine{}
      \def\elevator{$u_4$} \def\svgwidth{\columnwidth}
      \input{pics/Cessna-Plane.pdf_tex}
    \end{column}
    \begin{column}{0.35\textwidth}
      \begin{block}{Состояние, управление}%
        \begin{gather*}
          \vec{x}=\mathbf{x}=\langle x_1,x_2,\ldots,x_n\rangle,\\
          n=9,\\
          \vec{u}=\mathbf{u}=\langle u_1,u_2,\ldots,u_m\rangle,\\
          m=5,\\
        \end{gather*}
      \end{block}
    \end{column}
  \end{columns}
%\end{block}
\begin{block}{}
  \begin{gather}
x(t+1)=f(t,x(t),u(t)),\quad t \in T=\{t_0,t_0+1,...,t_1\},
	\label{eq100} \\
x(t_0)=x_0,\quad x(t)\in R^n,\; u(t) \in R^m,\quad t\in T,\label{eq100a} \\
I(x,u)=F(x(t_1))+ \sum_{t_0}^{t_1-1}f^0(t,x(t),u(t)) \to \min.
  \label{eq103}
  \end{gather}
\end{block}
\end{frame}
\begin{frame}{\textbf{Постановка задачи} }
\begin{block}{}
  \begin{gather}
x(t+1)=f(t,x(t),u(t)),\quad t \in T=\{t_0,t_0+1,...,t_1\},
	\label{eq100} \\
x(t_0)=x_0,\quad x(t)\in R^n,\; u(t) \in R^m,\quad t\in T,\label{eq100a} \\
I(x,u)=F(x(t_1))+ \sum_{t_0}^{t_1-1}f^0(t,x(t),u(t)) \to \min.
  \label{eq103}
  \end{gather}
\end{block}
\begin{block}{Задача улучшения}
  Пусть заданы управление $u^I(t)$ и соответствующее состояние $x^I(t)$. Требуется найти $x^{I\!I}(t),\; u^{I\!I}(t)$ такие, что
\[
I(x^{I\!I},u^{I\!I}) < I(x^I,u^I).
\]
 \end{block}
\end{frame}
% ---------------------------------------------------------
% ----------------------------------------------------------------
\begin{frame}{\textbf{Алгоритм первого порядка} }
\begin{block}{}
\begin{enumerate}
 \item[1.] Задается начальное управление $u^I(t)$, из уравнения \eqref{eq100} и условий \eqref{eq100a} определяется $x^I(t)$. Вычисляется $I(x^I,u^I).$
 \item[2.] Из системы $\psi(t)=H_x,\;\psi(t_1)=-F_x,$ находим $\psi(t),$\\ где $\psi(t) - n$ - вектор, $H(t,x,\psi(t+1),u)=\psi'(t+1)f(t,x,u)-f^{0}(t,x,u),\;$ производные функции $H$ находятся в точке $\left(t,x^{I}(t),\psi \left(t+1 \right),u^{I}(t)\right)$, <<$'$>> означает операцию  транспонирования.
 \item[3.] Из системы $x(t+1)=f(t,x(t),u^{I\!I}(t)), x(t_0)=x_0,$ где $u^{I\!I}(t)=u^{I}(t)+\alpha H_u,$ вычисляется $x^{I\!I}(t).$
 \item[4.] Новое управление и значение параметра $\alpha$ подсчитываются из решения задачи одномерной минимизации для функционала $I(x^{I\!I},u^{I\!I})\to \min\limits_{\alpha}.$
 \item[5.] Сравниваются значения функционалов $I\left(x^{I\!I},u^{I\!I}\right) \mbox{и}\quad
I\left(x^{I},u^{I}\right)$, если улучшения не произошло, то уменьшаем $\alpha$ и переходим к следующей итерации, начиная с пункта~3.
\end{enumerate}
\end{block}
\end{frame}
% ----------------------------------------------------------------

\begin{frame}{}
\begin{center}
\Huge\textbf{Спасибо за внимание!}
\end{center}

\end{frame}
% ----------------------------------------------------------------

\end{document}
%

%%% Local Variables:
%%% mode: latex
%%% TeX-master: t
%%% End:
