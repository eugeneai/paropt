\documentclass[10pt]{beamer}
\usepackage{amsmath,amssymb,longtable,hhline}
\usepackage{mathrsfs}
\usepackage{xcolor}

\usetheme{Warsaw}
\usecolortheme{crane}

\usepackage{iftex,ifxetex}
\ifPDFTeX
  \usepackage[utf8]{inputenc}
  \usepackage[T1]{fontenc}
  \usepackage[russian]{babel}
  \usepackage{lmodern}
  \usefonttheme{serif}
\else
  \ifluatex
    \usepackage{unicode-math}
    \defaultfontfeatures{Ligatures=TeX,Numbers=OldStyle}
    \setmathfont{Latin Modern Math}
    \setsansfont{Linux Biolinum O}
    \usefonttheme{professionalfonts}
  \fi
\fi

\begin{document}
\title{ПРИМЕНЕНИЕ КОМПЬЮТЕРНОЙ АЛГЕБРЫ\\
В РЕАЛИЗАЦИИ АЛГОРИТМОВ УЛУЧШЕНИЯ}
\author{Черкашин Е.А., Бадмацыренова С.Б.}
\institute[ИДСТУ СО РАН, ИРНИТУ]{\normalsize  ИДСТУ СО РАН, ИРНИТУ}
\date[2015]{<<Винеровские чтения --- 2015>>
\\[0.3cm]
16 -- 18 апреля 2015 г.,\\
Иркутск
}
%\date{\today}
\maketitle

% ----------------------------------------------------------------
\begin{frame}{\textbf{Постановка задачи} }
\begin{block}{}
  \begin{gather}
x(t+1)=f(t,x(t),u(t)),\quad t \in T=\{t_0,t_0+1,...,t_1\},
	\label{eq100} \\
x(t_0)=x_0,\quad x(t)\in R^n,\; u(t) \in R^m,\quad t\in T,\label{eq100a} \\
I(x,u)=F(x(t_1))+ \sum_{t_0}^{t_1-1}f^0(t,x(t),u(t)) \to \min.
  \label{eq103}
  \end{gather}
\end{block}
\begin{block}{Задача улучшения}
  Пусть заданы управление $u^I(t)$ и соответствующее состояние $x^I(t)$. Требуется найти $x^{I\!I}(t),\; u^{I\!I}(t)$ такие, что
\[
I(x^{I\!I},u^{I\!I}) < I(x^I,u^I).
\]
 \end{block}
\end{frame}
% ---------------------------------------------------------
\begin{frame}{}
\begin{center}
\Huge\textbf{Спасибо за внимание!}
\end{center}

\end{frame}
% ----------------------------------------------------------------

\end{document}
%

%%% Local Variables:
%%% mode: latex
%%% TeX-master: t
%%% End:
